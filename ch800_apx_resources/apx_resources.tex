% -*- mode: fundamental -*-

% ****************************************************************

\chapter{Resources: Documents and Tools}

\markboth{Ch \arabic{chapter}: Resources (DRAFT)}{\copyrightnotice}

\setcounter{page}{1}
\renewcommand{\thepage}{\Alph{chapter}-\arabic{page}}

\label{apx_resources}

% ****************************************************************

This appendix describes all the resources relevant to this course.

% ****************************************************************

\section{GitHub}

We will be using GitHub extensively.  Course materials will be
provided in a public GitHub repository, and GitHub's ``discussion''
facilities can be used to for questions and answers, visible to all.

For students who do not already know how to use GitHub, we will teach
the basics.

More detailed documentation can be found starting at:
\url{https://docs.github.com/en/get-started/quickstart}

% ****************************************************************

\section{RISC-V ISA (Instruction Set Architecture) Specifications}

\label{apx_resources_ISA_specs}

We will refer to the Unpriviledged ISA very frequently, so you may
wish to download a copy of the PDF for your laptop, and/or print a
copy.  The Privileged ISA document is not needed until later.

\begin{itemize}

\item ``The RISC-V Instruction Set Manual Volume I: Unprivileged ISA''.  

  Bibliography entry~\cite{RISCV_Unpriv_2019_12_13} contains a link
  to {\tt riscv.org} from which to download a PDF.

\item ``The RISC-V Instruction Set Manual Volume II: Privileged Architecture''

  Bibliography entry~\cite{RISCV_Priv_2021_12_03} contains a link
  to {\tt riscv.org} from which to download a PDF.

\end{itemize}

The \emph{formal specification} of the RISC-V ISA is written in the
Sail formal-specification language, and can be found at
\url{https://github.com/riscv/sail-riscv}.

\begin{itemize}

\item ``The RISC-V Instruction Set Manual Volume I: Unprivileged ISA''.  

  Bibliography entry~\cite{RISCV_Unpriv_2019_12_13} contains a link
  to {\tt riscv.org} from which to download a PDF.

\item ``The RISC-V Instruction Set Manual Volume II: Privileged Architecture''

  Bibliography entry~\cite{RISCV_Priv_2021_12_03} contains a link
  to {\tt riscv.org} from which to download a PDF.

\end{itemize}

% ****************************************************************

\section{RISC-V Trusted Simulators and Reference Programs for Testing Implementations}

\label{apx_resources_trusted_simulators}

The most well-known trusted simulator for RISC-V is the Spike
simulator, a free and open-source simulator that is written in C++ and
very carefully maintained by the RISC-V community as the standard
``reference model'' for RISC-V execution. The Spike simulator can be
found at: \url{https://github.com/riscv-software-src/riscv-isa-sim}

Another trusted simulator is the Sail model (the Sail model is the
official ``formal specification'' for RISC-V.  The Sail model and
simulator can be found at: \url{https://github.com/riscv/sail-riscv}.

Spike is usually more up-to-date with the latest ratified ISA
extensions, compared to the Sail model.

RISC-V International maintains a set of standardized tests that are
useful in testing new CPU implementations.  The following
repository---
\url{https://github.com/riscv-software-src/riscv-tests}--- contains
several hundred small test programs, written in RISC-V Assembly
Language, organized by ISA extension: RV32I, RV64I, A, M, F, D and C
extensions, Machine/Supervisor/User mode, {\etc}

% ****************************************************************

\section{RISC-V Assembly Language Manuals}

\label{apx_resources_asm_manuals}

We will not do very much assembly language programming, and we will
teach whatever notation we need during the course.

There are several RISC-V Assembly Language manuals available online,
and some in bookstores; download them only if you prefer a local copy:

\begin{itemize}

\item ``RISC-V Assembly Programmer's Manual'', 
    Palmer Dabbelt, Michael Clark and Alex Bradbury.

    Bibliography entry~\cite{Dabbelt2023} contains a link to online manual.

\item ``RISC-V ASSEMBLY LANGUAGE Programmer Manual Part I'', Shakti
    RISC-V Team, Indian Institute of Technology, Madras, India.
    Please see bibliography entry~\cite{Shakti_RISCV_ASM_Manual} for link
    from which to download a PDF.

\item ``An Introduction to Assembly Programming with RISC-V'', Edson Borin.

    Bibliography entry~\cite{Borin2021} contains a link
    from which to download a PDF.

\item ``RISC-V Assembly Language'',
    Anthony J. Dos Reis.

    Bibliography entry~\cite{DosReis2019}.  Available in bookstores.

\end{itemize}

% ****************************************************************

\section{RISC-V GNU tools, including {\tt riscv-gcc} compiler}

\label{apx_resources_gnu_tools}

We will be using the GNU tool chain, specifically the {\gcc} compiler
and linker, and the {\objdump} tool for disassembling an ELF file.

During the course we will show you how to install and use these tools.

The use of these tools is mostly the same as when targeting any target
architecture, including well-known architectures like x86 and ARM; the
student can find voluminous tutorial materials available on the GNU
tool chain on web and in books.

{\gcc} has some specific options for RISC-V; these are documented here:

\begin{tightlist}
  \item
  \url{https://gcc.gnu.org/onlinedocs/gcc/RISC-V-Options.html}

  \item
  \url{https://gcc.gnu.org/onlinedocs/gcc/gcc-command-options/machine-dependent-options/risc-v-options.html}
\end{tightlist}

It is also useful to know how to use the GNU debugger tool, {\gdb}.
Again, the student can find voluminous tutorial materials available on
on web and in books.

% ****************************************************************

\section{BSV}

In this course, we design the hardware of our RISC-V pipelined CPU
using the High Level Hardware Description Language {\BSV}.  The
reasons for our choice (instead of using Verilog, SystemVerilog or
VHDL) are discussed in more detail in Appendix~\ref{apx_Why_BSV} of
this document, as well as in the Introduction of the ``BSV by
Example'' book described below.

No advance knowledge of {\BSV} is needed for this course; we will
teach all necessary {\BSV} concepts during the course as we go along.

However, for those who would like to study {\BSV} on their own, or
wish to view additional {\BSV} materials, the following sections
provide some resources.

% ================================================================

\subsection{``BSV By Example'' book (free downloadable PDF)}

This book takes the student through a series of small, targeted {\BSV}:
examples:

\hm \emph{BSV by Example}, by Rishiyur S. Nikhil and Kathy R. Czeck, 2010.

Quoting from the Introduction:
\begin{quote}
``This book is intended to be a gentle introduction to BSV.''

`` This book tries to take you into the BSV language one small step at
a time. Each section includes a complete, executable (and
synthesizable) BSV program, and tries to focus on just one feature of
the language''
\end{quote}

A bound copy of the book can be purchased on Amazon, but a PDF copy of
the book and a tar file containing all the BSV program examples in the
book can be downloaded for free from the GitHub BSVLang repository at:

\url{https://github.com/BSVLang/Main/tree/master/Tutorials/BSV_Training}

\begin{tightlist}
  \item Book (PDF): \\
  \emph{repository}/\verb|Tutorials/BSV_by_Example_Book/bsv_by_example.pdf|

  \item Machine-readable version of all examples in the book: \\
  \emph{repository}/\verb|Tutorials/BSV_by_Example_Book/bsv_by_example_appendix.tar.gz|
\end{tightlist}

% ================================================================

\subsection{{\BSV} Tutorial}

A {\BSV} self-paced tutorial is available in the GitHub BSVLang repository:

\url{https://github.com/BSVLang/Main/tree/master/Tutorials/BSV_Training}

in the directory   \emph{repository}/\verb|Tutorials/BSV_Training/| which looks like this:

\begin{Verbatim}[frame=single]
BSV_Training/
    Build/
    Example_Programs/
        Common
        Eg02a_HelloWorld
        ...
        Eg03a_Bubblesort
        ...
        Eg04a_MicroArchs
        ...
        Eg05a_CRegs_Greater_Concurrency
        ...
        Eg06a_Mergesort
        ...
        Eg09a_AXI4_Stream
    Reference
\end{Verbatim}

Each of the \verb|Eg*| directories cotains a complete example, along
with documentation explaining the example, and instructions on how to
compile and Verilog-simulate it.  The \verb|Reference| directory
contains a collection of lecture slide decks explaining the {\BSV}
language.

% ================================================================

\subsection{MIT Course Material}

Massachusetts Institute of Technology (MIT) periodically teaches
courses on using {\BSV} for digital hardware design.  The following link:

\url{http://csg.csail.mit.edu/6.375/6_375_2013_www/handouts.html}

contains downloadable material:

\begin{tightlist}

  \item PDFs of slide decks for 12 lectures

  \item PDFs of slide decks for 4 tutorials classes

  \item PDFs and codes for 6 laboratories

\end{tightlist}

% ================================================================

\subsection{University of Cambridge Examples}

Prof. Simon Moore of University of Cambridge, UK, uses {\BSV} in his
teaching and research.  Several of his {\BSV} examples can be found here:

\url{https://www.cl.cam.ac.uk/~swm11/examples/bluespec/}

These examples are somewhat more advanced than the ones in the
previous sections.

% ================================================================

\subsection{{\bsc} download and installation; {\bsc} and {\BSV} manuals}

{\bsc} is free and open-source, and can be downloaded and installed as
described in {\BSV}'s GitHub web site
\url{https://github.com/B-Lang-org/bsc}.

On the main page of that repository you will find links to the
following documents (same links also given here):

\begin{itemize}

  \item The ``{\BSV} Language Reference
  Guide''~\cite{BSV_Lang_Ref_Guide}.  This document describes the
  syntax and semantics of {\BSV}.

  PDF: \url{https://github.com/B-Lang-org/bsc/releases/latest/download/BSV_lang_ref_guide.pdf}

  \item The ``BSC Libraries Reference
  Guide''~\cite{bsc_libs_ref_guide}.  This document describes the
  extensive set of libraries and IP (Intellectual Property blocks)
  available to the {\BSV} user.

  PDF: \url{https://github.com/B-Lang-org/bsc/releases/latest/download/bsc_libraries_ref_guide.pdf}

  \item The ``BSC User Guide''~\cite{bsc_user_guide}.  This document
  describes how to use the {\bsc} compiler, which compiles our
  hardware descriptions written in {\BSV} into Verilog (which can then
  be simulated or synthesizes using standard Verilog tools).

  PDF: \url{https://github.com/B-Lang-org/bsc/releases/latest/download/bsc_user_guide.pdf}

\end{itemize}

We will be using the Language Reference Guide and Librares Reference
Guide extensively, so you may wish to download a copy for your laptop.

% ****************************************************************

\section{Verilator (or other Verilog simulator)}

We will be doing Verilog simulations extensively during this course.
For low cost (free), and uniformity, we will be using Verilator.

During the course, we will show you how to install Verilator and use it.

The Verilator web site, \url{https://www.veripool.org/verilator/},
contains instructions on how to install Verilator, and also links to
PDF and HTML manuals for Verilator.  Version 5.004, or any more recent
version, will be suitable.

You can use other Verilog simulators if you prefer, but you should
independently know how to use them because we cannot offer support
during the course.  Some possibilities:

\begin{itemize}

  \item Icarus Verilog, also known as ``iverilog''.  This is a very
    good, free and open-source, easy-to-use Verilog simulator, but is
   quite slow compared to other Verilog simulators and so may be less
    useful for large designs.

    \url{https://steveicarus.github.io/iverilog/usage/getting_started.html}

  \item Commercial simulators from Synopsys, Cadence or Siemens/Mentor
    Graphics), Aldec, and others.  Each of these needs a paid license.

\end{itemize}

% ****************************************************************

\section{Amazon AWS}

\label{sec_AWS}

All hands-on work in this course will be run on the Amazon AWS cloud.
This way, everyone in the course has a common, stable, predictable
environment and we do not have to waste any time dealing with the
countless variations in environments found on different laptops and
servers.

During the course, we will explain all necessary concepts as we go
along, including how to set them AWS instances and use them.

The Amazon AWS cloud offers, on the ``AWS Marketplace'' a vast variety
of choices for virtual machines or, to use AWS terminology,
\emph{instances}.  We expect to use the following kinds of instances:

\begin{tightlist}

  \item[A:] An instance running the latest version of Ubuntu (Linux).

  \item[B:] A so-called ``F1 instance'', also running Ubuntu.  F1
    instances have attached FPGAs.

  \item[C:] An instance running the so-called ``AWS FPGA Developer
    AMI'' available in the AWS Marketplace.  This runs CentOS (Linux)
    and comes pre-installed with Xilinx Vivado tools, which we will
    use for creating FPGA bitfile images during the course.

\end{tightlist}

In Amazon's pricing, (B) is the most expensive, and so we will use
that only when we actually run on FPGA.  For general development and
simulation activities, we'll use (A) which is much cheaper.  We will
use (C) whenever we're creating a new FPGA bitfile image.

The standard Amazon documentation is can be found here:

\begin{tightlist}

  \item ``Set up to use Amazon EC2''

    \url{https://docs.aws.amazon.com/AWSEC2/latest/UserGuide/get-set-up-for-amazon-ec2.html}

  \item ``Tutorial: Get started with Amazon EC2 Linux instances''

    \url{https://docs.aws.amazon.com/AWSEC2/latest/UserGuide/EC2_GetStarted.html}

\end{tightlist}

% ****************************************************************

\section{Xilinx Vivado}

The FPGAs on Amazon AWS F1 instances are Xilinx Ultrascale FPGAs.
Thus, when we build bitfiles on Amazon AWS, we will be using Xilinx
Vivado tools (which are provided by AWS for zero incremental cost on
AWS FPGA Developer AMI instances, see \ref{sec_AWS}).

During this course, we will explain all necessary concepts as we go
along.

When building a bitfile, it is particularly useful to understand how
to interpret the Vivado timing and resource reports.  The timing
report indicates:

\begin{tightlist}

  \item whether or not our design has successfully met our desired
    frequency target (MHz), and

  \item if it did not, which part of our circuit is the likely
  culprit, which needs to be fixed.

\end{tightlist}

The resource report indicates the ``size'' or our design (how may
LUTs, flip-flops, BRAMs, DSPs, {\etc}).

For more details, Xilinx has extensive documentation for which a good
starting point is the ``Vivado Design Suite Overview'' at

\url{https://docs.xilinx.com/r/en-US/ug910-vivado-getting-started/Vivado-Design-Suite-Overview}.

% ****************************************************************

\section{RISC-V textbooks}

\label{apx_resources_RISCV_textbooks}

This course is self-contained, and it is not necessary to acquire any
textbooks.

The following list is provide only as a courtesy and convenience.  All
these books are written using the RISC-V instruction set as examples,
and are available in bookstores.

\begin{itemize}

  \item ``The RISC-V Reader: An Open Architecture Atlas'',
    David Patterson and Andrew Waterman,
    Strawberry Canyon, 2017.  Available in bookstores.

    Bibliography entry~\cite{PattersonWaterman2017}.

  \item
    ``Computer Organization and Design RISC-V Edition (2nd Edition): The Hardware Software Interface'',
    David A. Patterson and John L. Hennessy
    Morgan Kaufman, 2020. Available in bookstores.

    Bibliography entry~\cite{PattersonHennessy2020}.

  \item ``Computer Architecture: A Quantitative Approach, 6th Edition'',
    John L. Hennessy and David A. Patterson,
    Morgan Kaufmann, 2017.  Available in bookstores.

    Bibliography entry~\cite{Hennessy2017}.
    This is the ``classic'' textbook on computer architecture, a more
    advanced textbook.

\end{itemize}

% ****************************************************************
