% -*- mode: fundamental -*-

% ****************************************************************

\chapter{BSV: Rules and Methods I: Clocks and Rule Scheduling}

\markboth{Ch \arabic{chapter}: BSV: Rules I}{\copyrightnotice}

\setcounter{page}{1}
% \renewcommand{\thepage}{\arabic{page}}
\renewcommand{\thepage}{\arabic{chapter}-\arabic{page}}

\label{ch_Rules_I}

% ****************************************************************

% ----------------
\vspace{2ex}

\centerline{\includegraphics[width=1in,angle=0]{Figures/Fig_Under_Construction}}

\vspace{2ex}
% ----------------

\section{Introduction}

Rules are the fundamental dynamic behavior constructs in BSV.  A rule
is an infinite process.  Every rule consists of a \emph{condition} and
an \emph{action}: whenever the condition is true, the action is
performed; we say the rule ``fires'' whenever its condition is true.
A rule may not fire even if its condition is true, if it ``conflicts''
with another rule.

A rule may contain explicit and implicit conditions. 

Every method has an implicit conditions.  For a FIFO, the \emph{enq()}
method's implicit condition is false when the FIFO is full, {\ie} when
it does not contain space to enqueue a new item.  We also say the
method is ``enabled'' when its implicit condition is true.

\section{Rules, Actions and clocks}

\section{Rule semantics}

% ****************************************************************
