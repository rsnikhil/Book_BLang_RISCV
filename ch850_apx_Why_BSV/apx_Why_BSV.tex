% -*- mode: fundamental -*-

% ****************************************************************

\chapter{Why BSV?}

\markboth{Ch \arabic{chapter}: Why BSV? (DRAFT)}{\copyrightnotice}

\setcounter{page}{1}
\renewcommand{\thepage}{\Alph{chapter}-\arabic{page}}

\label{apx_Why_BSV}

% ****************************************************************

The BSV language is a modern, high-level, hardware description
language with a strong formal semantic basis.  It is \emph{fully
synthesizable}, {\ie} the \emph{bsc} compiler can also compile your
source code into Verilog \cite{IEEEVerilog2005a}, which we regard as
the ``assembly language'' of hardware design.  That Verilog code can
then be further compiled by ASIC synthesis tools such as Synopsys'
Design Compiler or FPGA synthesis tools such as Xilinx Vivado or
Altera Quartus, for implementation in ASICs and FPGAs, respectively.

BSV is very suitable for describing architectures precisely and
succinctly, and has all the conveniences of modern advanced
programming languages such as expressive user-defined types, strong
type checking, polymorphism, object orientation and even higher order
functions during static elaboration.

All computation in BSV is expressed using ``Rules''.  For many people,
this takes a little acclimatization because it is \emph{very}
different from traditional programming models (such as in C++ or Java)
which are based on sequential processes.  But over time, it becomes
\emph{the} natural way to think about hardware computation, which is
based on massive, fine-grained, heterogeneous parallelism.  Complex
and high-speed hardware designs are full of very subtle issues of
concurrency and ordering, and BSV's computational model is one of the
best vehicles with which to study and understand this.

Modern hardware systems-on-a-chip (SoCs) have so much hardware on a
single chip that it is useful to conceptualize them, analyze them and
design them as \emph{distributed systems} rather than as globally
synchronous systems (the traditional view), {\ie} where architectural
components are loosely coupled and communicate with messages, instead
of attempting instantaneous access to global state.  This is because
the delay in communicating a signal across a chip is now comparable to
the clock periods of individual modules.  Again, BSV's computational
model is well suited to this style of design.

A key to architecting complex systems and reusable modules, whether in
software or in hardware, is powerful \emph{interfaces}.  Module
interfaces in BSV are object-oriented (based on methods),
polymorphic/generic, and capture certain computational protocols.
This facilitates creating highly reusable modules, enables quick
experimentation with alternatives structures, and allows designs to be
changed gracefully over time as the requirements and specifications
evolve.

Architectural models written in BSV are fully executable.  They can be
simulated in the Bluesim$^{\rm TM}$ simulator; they can be synthesized
to Verilog and simulated on a Verilog simulator; and they can be
further synthesized to run on FPGAs or be etched into ASIC silicon, as
illustrated in Fig.~\ref{Fig_Topics}.  Even when the final target
is an ASIC, the ability to run on FPGAs enables early architectural
exploration, early development of the software that will later run on
the ASIC, and much more extensive and early verification of the
correctness of the design.  Students are also very excited to see
their designs actually running on real FPGA hardware.

In this book, we teach the use of BSV for the design of complex
hardware modules and systems by going in detail through a series of
examples, and exploring basic concepts as needed along the way, such
as combinational circuits, pipelines, data types, modularity and
complex concurrency.  At every stage the student is encouraged to run
the designs at least in simulation, but preferably also on FPGAs.

By the end of the course we will have seen all the source code for a
complete simple, pipelined RISC-V CPU along with a small ``SoC''
(System-on-a-Chip) including an interconnect, a connection to memory,
and a few devices such as a UART.

\hspace{1cm}

{\large\bf Who uses BSV?}

BSV has been used in teaching and research at major universities,
including MIT (Massachusetts Institute of Technology, USA),
Universisty of Cambridge (UK), Indian Institute of Technology, Madras
(India), Indian Institute of Technology, Mumbai (India), Seoul
National University (South Korea), University of Texas at Austin
(USA), Carnegie Mellon Univerity (USA), Georgia Institute of
Technology (USA), Cornell University (USA) and Technical University of
Darmstadt (Germany).

BSV has been used to design major IP components in commercial ASICs
from Texas Instruments, ST Microelectronics and Google.  It has been
used for FPGA-based modeling at IBM, Intel, Qualcomm, Microsoft
Research, several DARPA projects, and others.  It is being used for
commercial RISC-V processors from InCore Semiconductors (Shakti line
of RISC-V processors, India) and The C-DAC (Center for Development of
Advance Computing) (Vega line of RISC-V processors, India).

% ****************************************************************

\section{Why BSV instead of some other Hardware Design Language?}

\begin{center}\fbox{
\begin{minipage}{\hlessmm}
\emph{The rest of this chapter is intended for those interested in comparing
BSV's approach to other approaches (Verilog, SystemVerilog, VHDL, and
SystemC), and can be safely skipped by others who just want to get on
with learning BSV.}
\end{minipage}}
\end{center}

\noindent
One may be curious why the material in this book could not have been
covered using one of the more widely known languages for hardware
design: Verilog~\cite{IEEEVerilog2005a}, VHDL~\cite{IEEEVHDL2002},
SystemVerilog~\cite{IEEESystemVerilog2012a},
SystemC~\cite{IEEESystemC2011a}.  There are several reasons, outlined
below.

In the following paragraphs, we will refer to all the above languages,
or at least their synthesizable subsets, as ``RTL'' (Register Transfer
Level languages).

% ================================================================

\subsection{A better computational model}

Paradoxically, the formal definitions of the semantics of traditional
hardware design languages (HDLs)--- Verilog, SystemVerilog, VHDL and
SystemC--- are not in terms of hardware concepts, but in terms of
\emph{software simulation} on conventional computers.  Like
traditional software programming languages, they are defined in terms
of sequential statement execution, with traditional conditionals,
loops, and procedure calls and returns, reading and writing
conventional variables.  Programs can have multiple concurrent
\emph{processes} (e.g., ``always blocks'' in Verilog), but each of
them is defined with traditional sequential programming semantics.

Digital hardware, on the other hand, has a quite different computation
model.  It consists of hundreds, if not thousands of concurrent
``state machines'' that transform the current state of the hardware,
implemented using registers, memories and FIFOs.  By and large, there
is no sequencing of these state machines based on program counters or
statement sequences.  Rather, these state machines are independent and
``reactive'', {\ie} each one performs an action whenever certain
conditions hold, e.g., when a register holds a particular value, or a
value is available in a FIFO, etc.

To bridge this rather large gap from conventional sequential processes
to concurrent reactive state machines requires a major mental shift.
One must severely restrict code to only a much smaller so-called
\emph{synthesizable subset} of a conventional HDL.  Processes are
restricted to simple clocked loops: ``\verb|always @posedge CLK ...|'',
also known as an ``always-block''.  Even more draconian is a
transposition away from the natural concurrent state machine view to a
\emph{state element-centric} view: even though a state element may be
read and written by multiple state machines, all updates to that state
element must be concentrated in a single always-block, usually in a
large conditional construct (if-then-else, case, ...) that describes
all the different contributions of different state machines.  This
transposition, from the natural state machine-centric view to the
rather unnatural state element-centric view, is necessary because in
the the synthesizable subset there is no synchronization between
always-blocks; the programmer has to plan every detail of how to
resolve (arbitrate) competing updates to each state element.

In other words, in conventional HDLs, neither the simulation view
(sequential processes) nor the synthesizable view (state
element-centric always blocks) are a natural way to model hardware
behavior.

BSV programs, instead, directly express the natural model of
hardware--- concurrent state machines.  Each ``rule'' in BSV is a
reactive state transition that awaits some condition on the hardware
state and then takes an action to transforms the state.  Further, each
rule is an \emph{atomic transaction}, {\ie} the details of how one
arbitrates competing accesses from multiple rules to common shared
state is left to the compiler.  This kind of arbitration logic, which
is hand-written in other HDLs, is a major source of bugs.

In BSV, unlike in other HDLs, the semantics are identical whether you
execute in simulation or in hardware---there is no mental gear shift
necessary, and simulation behavior is always identical to synthesized
hardware behavior.

Finally, the Rules computation model uniquely encourages
\emph{refinement}, a powerful design methodology.  We initially create
a high-level, approximate model of a target design, using a few large
rules.  Both the level of micro-architectural detail and the range of
functionality are approximated (abstracted).  Often such a model can
be written in less than a day, and it can immediately be executed to
verify functional correctness.  Then, over time, we incrementally add
architectural detail---for example, pipeline registers and state
machines with more, smaller steps---and the original rules (large step
state transitions) are replaced by more, smaller rules (small step
state transitions).  The atomic semantics of rules makes this a robust
methodology, {\ie}, a refinement does not have a large ripple effect.
This is in quite dramatic contrast to the difficulty in changing RTL,
which is notoriously brittle and unforgiving.

Refinement allows early and continous confidence in functional
correctness and completeness, since we execute the code very
frequently.  Refinement allows mid-course corrections in
functionality, after observing execution on real data.  Refinement
allows separating \emph{functionality} from \emph{performance},
achieving functionality early and holding it constant while we improve
performance to meet a performance target (by target performance we
mean some desired targets for speed, area, and power).

% ================================================================

\subsection{Modern language features}

The field of programming languages has seen tremendous progress since
the early days (1950s).  Modern high-level languages have advanced
type systems (polymorphism, typeclasses and overloading, functional
types, and so on).  Modern high-level languages have strong mechanisms
for encapsulation and abstraction (such as object-orientation) which
promote the separation of concerns between externally visible behavior
and internal representation choices.  Modern high-level languages make
frequent use of higher-order functions---functions whose arguments and
results can themselves be functions and data structures whose
components can be functions.

Unfortunately, practically none of these powerful features are present
in the synthesizable subsets of conventional
HDLs\footnote{SystemVerilog and SystemC have object-orientation,
polymorphism, and overloading, but these are typically used only in
simulation for verification environments of hardware designs, not for
actual hardware design itself.}.  BSV, on the other hand, adopts the
full power of the Haskell functional programming
language~\cite{PeytonJones2003}: algebraic types, functional types,
polymorphic types, typeclasses, higher-order functions, and recursive
and monadic static elaboration.  This delivers unprecedented
expressive power, type safety and type flexibility in a hardware
design language.

% ================================================================

\subsection{Comparison with C++-based High Level Synthesis}

\label{apx_HLS}

Recently, some tools have become available under the rubric of ``High
Level Synthesis'' (HLS) that claim to shield you from this mental gear
shift from simulation to hardware.  Designs are written in a
traditional sequential programming language (typically C++), and an
HLS tool automatically compiles this into a hardware implementation.
While beautiful in concept, there are many serious limitations in
practice, which are discussed below.

% ----------------------------------------------------------------

\subsubsection{C++ codes need significant rewriting}

C++ HLS tools will rarely accept arbitrary, off-the-shelf C++ codes
and produce good hardware implementations.  C++ codes often requre
significant restructuring to achieve good results.

First, the tools only accept a limited subset of C++ syntax. In
particular, these tools are very averse to any kind of pointer-based
argument passing or data structures, unless all the pointers can be
resolved by the compiler ({\ie}, the compiler statically knows the
addresses represented by the pointers).  This is because, while C++
normally executes on machines that provide the abstraction of a single
large memory with a single address space (so a pointer is
fundamentally an address, and dynamic allocation and relocation are
easy), hardware designs typically use hundreds or thousands of
individual memory units, from registers to register files to SRAMs,
DRAMs, Flash memories, and ROMs, each with its own address space.

Second, most C++ codes written for conventional execution rely deeply
on sequential execution.  For example, they may re-use a variable
(multiple reads and writes in different phases of the code).  Many of
these programming techniques, often a good idea for higher performance
and smaller memory footprints in conventional execution, are exactly
the opposite of what is needed for hardware implementation, which is
highly parallel.

Overall, for good results, one must develop a keen sense of the
hardware implementation impact of various ``styles'' of writing C++
code.  Small changes in style can mean the difference between a
terrible implementation and an acceptable one.  One vendor insists
that any team adopting their tool should not consist solely of C++
experts, but must also include hardware engineers.

% ----------------------------------------------------------------

\subsubsection{Narrow range of applicability due to automatic parallelization}

C++ is, by official definition, a completely sequential language.
Hardware, on the other hand, relies on massive, fine-grain
parallelism.  It is the HLS tool that has to pull off this magical
transformation.

C++ HLS tools rely on a body of knowledge called CDFG Analysis
(Control and Data Flow Graph Analysis).  After parsing and
typechecking, the C++ program is represented internally in a data
structure called the CDFG. This CDFG, initially directly reflecting
the sequential nature of the source, is analyzed and transformed into
a parallel representation from which, eventually, hardware is
generated.

It turns out that this transformation only works well for a narrow
range of program structures---cleanly nested \verb|for|-loops with
fixed iteration bounds, operating on dense rectangular arrays.  Of
course, many signal-processing and image-processing applications do
have this structure, and C++ HLS tools have found their greatest
success in this arena.

But the moment we step outside this sweet spot, towards sparse arrays
or programs that are highly control-dominated, these tools fall off a
cliff.  Most hardware design in fact involves components that don't
fall into the C++ HLS sweet spot: CPUs, cache systems, switched
interconnects, flash memory and disk controllers, high-speed I/O
controllers for Ethernet, PCIe, USB, and so on.  For example, we are
unaware of any project using C++ HLS for CPU design, whereas there are
over a dozen such projects using BSV.

% ----------------------------------------------------------------

\subsubsection{Lack of ``Algotecture'': Architectural transparency and predictability}

Most people with some training in Computer Science are familiar with
the idea that Algorithms are Job One---when writing
performance-critical software, the first-order concern is to design a
good algorithm.  Further, creating a good algorithm is a creative act;
compilers don't automatically create good algorithms for
you\footnote{Of course, there is research in this area, but this
starts entering the realm of Artificial Intelligence.}.

Unfortunately, because most of our codes run on classical von Neumann
machines, many people forget that, when the execution platform
changes, our old algorithms may no longer be any good---the
assumptions about the cost of fundamental operations may no longer
valid and in fact may be wildly different, requiring a complete
re-think of the algorithm.

This bring up a fundamental difference between software design and
hardware design.  In software, you are given a particular target
architecture (CPU, GPU, cluster, vector machine, ...), and the
designer's job is to design a good algorithm for that fixed
architecture.  In hardware, on the other hand, the designer's job is
to design the algorithm and the architecture \emph{jointly}.  In other
words, for hardware designers, algorithm and architecture are joined
at the hip; it is meaningless to separate these activities.  We thus
use the term \emph{Algotecture} to describe this integrated activity.

Unfortunately, most C++ HLS tools provide very narrow visibility and
control into architecture.  For example, directives for loop unrolling
and loop fusion may allow you to express some variation in iterative
{\vs} parallel {\vs} pipelined structures.  But, basically, it's the
tool that chooses the architecture, and you have some weak knobs to
guide its choices.  A common syndrome with C++ HLS tools is that one
quickly produces an implementation, but it is terrible in area or
performance, and this is followed by a \emph{long} tail of activity in
which the designer tweaks the knobs every which way in an effor to
beat it down into the desired performance envelope.

In contrast, with BSV, architectural choices (like algorithmic
choices) are in the hands of the designer, where it should be.  There
are no surprises with respect to architecture; performance is never a
mystery, and the designer can quickly improve it and converge to an
acceptable solution.

% ----------------------------------------------------------------

\subsubsection{Summary}

In summary, it is our experience that BSV is a much better language
for complex hardware design, whether control or data oriented, whether
for modeling or architectural exploration or final implementation, or
for synthesizable on-FPGA verification transactors.  Following the
philosophy of DSLs (Domain Specific Languages), BSV is very much an
expressive DSL beautifully suited for hardware design, whereas
sequential C++ is certainly not (it was never intended to be!).

% ****************************************************************
