% -*- mode: fundamental -*-

% ****************************************************************

\chapter{BSV: Importing C/C++ functions into BSV simulations}

\markboth{Ch \arabic{chapter}: BSV: Importing C}{\copyrightnotice}

\setcounter{page}{1}
\renewcommand{\thepage}{\Alph{chapter}-\arabic{page}}

\label{apx_BSV_importing_C}

\label{Sec_Importing_C}

% ****************************************************************

\section{Introduction}

\index[BSV]{Importing C and C++ functions}

This facility is only used in BSV simulation (compiling C code to
hardware is very difficult even in limited contexts, see
Section\ref{apx_HLS}) and it is principally used in \emph{testbenches}
for designs.

There are many reasons to import C functions into BSV (whether
simulating in Bluesim or in Verilog simulation):

\begin{itemize}

 \item Many applications begin life as a C/C++ model, such as a C/C++
       algorithm that we are trying to accelerate in hardware, or an
       algorithm that we are still prototyping in C/C++ while
       developing the rest of the system.

 \item C/C++ is useful for testbench components: They will run much
       faster than Verilog simulation, and so will not be a
       performance burden on simulating the Verilog of interest.  They
       have full-service access to data files and operating-system
       services.

 \item Even for an actual hardware component for which we have BSV or
       Verilog code, we may temporarily substitute it with a much
       faster C model while our focus is on testing other parts of the
       hardware.

\end{itemize}

In Drum and Fife code, the whole CPU design is written in BSV.
However, to test the CPU we need to connect it to a memory system, a
UART, {\etc} and for these we use models written in C.  Modeling the
memory system in C makes it easy to include code for pre-loading ELF
and memhex files.

A C function can be imported into BSV with the simple steps described
in the next sections.

BSV can only import C functions directly.  If you need to use a C++
function, write a C wrapper function that is invoked from BSV; that C
function can then invoke your C++ function.

An imported C function invoked in BSV is semantically instantaneous,
never a temporal process.  It is imported with \verb|Action| or
\verb|ActionValue| type or a pure (combinational) type.\footnote{Of
course an invoked C function, before it returns, could start a C
``pthread'' which then runs concurrently with the BSV simulation.}

% ****************************************************************

\section{In BSV code, declare a BSV version of the C function}

\index[BSV]{import "BDPI"@{\tt import "BDPI"}}

In the BSV code, the imported C function will be invoked exactly like
a normal BSV function.  We declare the header of this ``BSV'' function
({\ie} just the result type, function name and arguments with their
types, and not the function body), preceded by the phrase {\tt import
"BDPI"}:

\begin{center}
 \fbox{\small
  \begin{minipage}{5in}
   \begin{tabbing}
   {\tt import "BDPI"} \\
   {\tt function} \hm \emph{BSV-type} \hm \emph{function-name (} \= \emph{BSV-type \hm arg} \\
                                                                 \> ... \\
                                                                 \> \emph{BSV-type \hm arg} {\tt );}
   \end{tabbing}
  \end{minipage}
 }
\end{center}

The argument- and result-passing conventions are simple, adjusting for
the C's limitation that arguments and results be can at most 64-bits
wide:

\begin{itemize}

 \item BSV values of sizes up to 8, 16, 32 or 64 bits are passed as
       arguments and results of C type \verb|uint8_t|,
       \verb|uint16_t|, \verb|uint32_t| and \verb|uint64_t|,
       respectively, for both arguments and results, in the same
       corresponding positions.

 \item For a BSV argument value of size greater than 64 bits, it is
       passed to C as a pointer to memory containing the BSV value.

 \item For a BSV \emph{result value} of size greater than 64 bits, the
       corresponding C function prototype changes slightly:

       \begin{tightlist}

        \item It gains an extra first argument which gets a pointer to
              memory which should be filled by the C function with the
              BSV value to be returned.

        \item Because of this, it has a \verb|void| return type.

       \end{tightlist}
\end{itemize}

% ****************************************************************

\section{Compile the BSV code with the \emph{bsc} compiler}

If you are compiling for Bluesim, there is no change to how you invoke
\emph{bsc}.

If you are compiling to Verilog, provide the additional flag
``\verb|-use-dpi|'' on the command line.  This will generate standard
SystemVerilog \verb|import "DPI-C"| declarations in the generated
Verilog.

% ****************************************************************

\section{Linking}

For Bluesim linking, simply provide the C file(s) implementing
imported C functions as an additional command-line arguments when you
again invoke \emph{bsc} for linking.

For Verilog linking, in the imported C code we recommend adding the
following for each of the imported C functions:

{\small
\begin{Verbatim}[frame=single, numbers=left]
#ifdef __cplusplus
// 'C' linkage is necessary for linking with Verilator object files
extern "C" {
    ... imported C function's prototype ...
}
#endif
\end{Verbatim}
}

This directs the C/C++ compiler to compile the C function with C
argument-passing conventions instead of C++ argument passing, which is
necessary for SystemVerilog DPI-C.  Verilog simulator tools like
Verilator compile the C code with a C++ compiler which, by default,
would use C++ argument-passing.

For Verilator linking, simply provide the C/C++ file(s) implementing
the imported C functions as additional arguments to the
\verb|verilator| command.

For other Verilog simulators, the linking details may vary; please
consult their respective manuals or experts on how to link-in C code
under the SystemVerilog DPI-C standard.

% ****************************************************************

\section{Recommendations for arguments and results of imported C/C++ functions}

The following recommendations circumvent potential complications when
using imported C code in BSV.

% ================================================================

\subsection{Only use BSV types corresponding to C types}

The data passed between BSV and C have the standard BSV packed
representations in bits.  It can be tricky in C to deal with, say, a
BSV 13-bit value that, in C, is a \verb|uint16_t|.  We recommend only
using BSV arguments/results whose sizes are multiples of 8-bits so
that they map exactly to bytes in C.

Even for BSV structs and vectors, only use structs and vectors whose
elements are 8-bit byte-sized.  Accessing/updating the components in C
is vastly easier with these constraints.

If an original BSV type $T$ does not have ``byte-aligned'' components,
we often define a new type $T2$ with byte-aligned components and copy
values from $T$ to $T2$ before the call (for arguments) or from $T2$
to $T$ (for results).  These is extra, technically unnecessary
``copying'' of data, but the resulting simplification of the C code is
well worth it.

For structs, be careful that C structs may have ``gaps'' or
``padding'' between fields to improve word-alignment, whereas BSV
structs are tightly packed.  Thus a BSV struct and a C struct, though
they may look identical, may have different data representations.

% ================================================================

\subsection{Use {\tt ActionValue\#(t)} for imported C function's result}

In an \verb|import "BDPI"| declaration the type of the result the
function may be \verb|Action|, or \verb|ActionValue#(t)| or some type
$t$.  As discussed in Section~\ref{Sec_Pure_vs_Side_Effect_functions},
in BSV the last case (not an action or actionvalue) is taken as a
strong guarantee of mathematical purity (and absence of side-effects);
the \verb|bsc| compiler may merge multiple invocations that have the
same arguments, and it may move a function elsewhere in the code with
different control conditions.  These optimizations can lead to nasty
surprises when the C function is not really pure (including core-dumps
if it relies on prior initializations).

We recommend normally using \verb|ActionValue#(t)| for the result of
any imported C function (or \verb|Action| if the C result is
\verb|void|) to avoid surprises.

Use a non-action/actionvalue return type \emph{only} if you are
absolutely sure that the C function is pure and does not need any
other initializations before invocation ({\eg} the C library
\verb|toupper| character transfomer or \verb|cos()| trigonometric
function).  Even for these, be safe and use an actionvalue function.

% ****************************************************************

\section{Example: Memory Model for Drum and Fife}

For Fife and Drum, because our focus is on the CPU module, we
implement the memory system in C for convenience.  In addition to the
speed reason (not slowing down simulation), C is also more convenient
for reading in memhex32 and ELF files.

\input{Code_Extracts/Mems_Devices}

The corresponding C function prototype, in a C file is:

{\small
\begin{Verbatim}[frame=single, numbers=left]
#ifdef __cplusplus
// 'C' linkage is necessary for linking with Verilator object files
extern "C" {
void c_mems_devices_req_rsp (uint8_t        *result_p,
			     const uint64_t  inum,
			     const uint32_t  req_type,
			     const uint32_t  req_size_code,
			     const uint64_t  addr,
			     const uint32_t  client,
			     uint8_t        *wdata_p);
}
#endif
\end{Verbatim}
}

% ****************************************************************
