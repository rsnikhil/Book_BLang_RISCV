% -*- mode: fundamental -*-

% ****************************************************************

\chapter{Glossary}

\markboth{Ch \arabic{chapter}: BSV (DRAFT)}{\copyrightnotice}

\setcounter{page}{1}
\renewcommand{\thepage}{\Alph{chapter}-\arabic{page}}

\label{apx_Glossary}

% ****************************************************************

\begin{itemize}

\item[\bf ASIC] Application-Specific Integrated Circuit. A kind of
  electronic device that represents a desired digital circuit directly
  in silicon and has been fabricated for that purpose (not
  customizable and general-purpose like an FPGA).

\item[\bf API] Application Programming Interface.  Term commonly used
  in many programming languages, methodologies and protocols to
  describe the set of functions/procedures/methods used to interact
  with a module/object by external entitities (from outside the
  module/object).  The API clearly separates external concerns from
  internal concerns.  External concerns are about ``what'' a method
  does or sequence of methods do: what are their argument and result
  types, and what do they (abstractly) achieve.  Internal concerns are
  about ``how'' methods do what they are supposed to do.  This
  separation of concerns also allow transparently substituting a
  module implementation with an alternate implementation ({\eg} for
  greater efficiency) without disturbing the external context.

\item[\bf BSV, BH] An open-source, modern, High-Level HDL.  Two
  optional syntaxes (choose to one's taste): BSV has traditional
  Verilog-like syntax, BH has traditional Haskell-like syntax.

\item[\bf CPU] Central Processing Unit.  The computational element of
  a computer.

\item[\bf CSRs] Control and Status Registers.  These are special
  registers in the ISA, most of which are accessibly only while
  executing at higher privilege levels (Machine and Supervisor).
  Certain key CSRs play a central role in disciplined transition
  between privilege levels, in virtual memory, and in memory
  protection.

\item[\bf DRAM] Dynamic Random Access Memory.  A kind of silicon chip
  that implements memory.  Compared to SRAM, is larger (number of
  bits), denser (bits per silicon area), cheaper (\$ per bit), uses
  less power (watts per bit) and is more complex to operate (needing
  regular refreshing {\etc}). Usually off-chip (not part of an ASIC or
  an FPGA).

\item[\bf FPGA] Field Programmable Gate Array.  A kind of electronic
  device that has configurable circuits that can be customized to
  represent any desired digital design.  These are catalog parts
  available from several vendors.

\item[\bf FPGA Board] A circuit board containing one or more FPGAs, a
  power supply, and DRAM memories.  Often contains other facilities
  such as GPIO, UARTs, JTAG, PCIe bus connections, Ethernet
  connection, USB connection, Flash memory, and so on.

\item[\bf FSM] Finite State Machine.  A sequential process that moves
  (``transitions'') from one state to another in a fixed repertoire of
  states.  Transitions may loop back to earlier states, and may
  conditionally select one of a set of alternative next-states.

\item[\bf GPIO] General Purpose Input Output.  An electronic device
  attached to a computer system. When the CPU stores a byte/word to a
  GPIO address, the bits of the word appear as electronic signals from
  the device, and can be used as an \emph{actuator}---switch on/off a
  back of LED lamps, a relay, a motor, {\etc.}.  When the CPU loads a
  byte/word from a GPIO address, it can read the state of a
  \emph{sensor}---switches, photocells, motor speed, temperature,
  {\etc.}

\item[\bf GPR] General Purpose Register.  For RISC-V, just a synonym
  for the basic register set holding integers.  They are ``general
  purpose'' in the sense that software is free to use them in any way
  (in contrast with some earlier ISAs that restricted certain
  registers to certain roles, such as holding addresses).

\item[\bf HDL] Hardware Design Language.  A language in which one can
  represent circuits, and for which there are tools that can render a
  program into actual circuits for FPGAs and ASICs.  Examples include:
  BSV, BH, Chisel, Verilog, SystemVerilog, VHDL.

\item[\bf HLHDL] High-Level Hardware Design Language.  An HDL with
  higher-levels of abstraction and more powerful constructs and
  semantics compared to the traditional HDLs Verilog, SystemVerilog
  and VHDL, in the same sense that modern software programming
  languages (Java, Python, Javascript, Haskell, OCaml, ...) have
  higher-levels of abstraction than C/C++ which, in turn, have higher
  levels of abstraction than Assembly Language.  Examples include BSV,
  BH (the Haskell-syntax variant of BSV), Chisel, and HLS.

\item[\bf HLS] High Level Synthesis.  The term typically used for
  tools and methodology that compile C/C++/SystemC programs into
  hardware.  HLS can be fragile in that it works best only on certain
  subsets of C/C++ (``simple rectangular loop and array'' algorithms),
  and require certain coding styles and directives.

\item[\bf ISA] Instruction Set Architecture.  A specification of
  instructions: how an instruction is coded in bits; ``architectural
  state'' (PC, registers {\etc}); what it means to execute an
  instruction; assembly language syntax.  The specification is
  described independently of any particular implementation,
  traditionally in a manual with text and diagrams, occasionally and
  recently also in a formal-specification language.

  An ISA can (and typically does) have many possible implementations,
  varying widely in speed, size, power, cost, technology (ASIC, FPGA),
  {\etc} Examples of famous ISAs and vendors who supply
  implementations include RISC-V (diverse vendors), x86 (Intel and
  AMD), ARM (Arm, Apple, Samsung, others), Sparc (Sun, Oracle,
  Fujitsu, others), MIPS (MIPS, Inc.), Power and PowerPC (IBM,
  others), ...

\item[\bf Microarchitecture] The structural and behavioral details of
  an ISA implementation that are \emph{below} the level of abstraction
  of the ISA, {\ie} not demanded by the ISA but chosen by the
  implementor for practical reasons (speed, power, area, cost, ...).
  Examples: pipelines, branch prediction, scoreboards, register
  renaming, out-of-order execution, superscalarity, instruction
  fission and fusion, replicated execution units, store-buffers, ...

\item[\bf OS] Operating System.  Can vary from small, embedded,
  real-time OSs such as FreeRTOS, to more capable embedded OSs like
  Zephyr, to secure micro-kernels like seL4, to full-featured OSs like
  Linux, Windows, MacOS, Solaris, AIX, {\etc}

\item[\bf RISC-V] A particular standard ISA.  Originated circa
  2008-2010 in research at University of California, Berkeley, and
  subsequently spun out (2010s) into an international non-profit
  consortium ``RISC-V International'' (RVI) headquartered in
  Switzerland (\url{https://riscv.org}).

  Unlike other well-known ISAs, the RISC-V ISA is an \emph{open}
  standard, {\ie} implementors do not need to pay any license fee in
  order to use the ISA, which is one of the factors behind its wide
  adoption by hundreds of vendors.

\item[\bf RTL] Register-Transfer Level/Language.  This is a level of
  abstraction of describing hardware that assumes that the available
  primitive components are clocked registers and combinational
  circuits for multiplexers, and basic arithmetic and logic functions
  (adders, subtractors, boolean operations, shifters, {\etc}).

  This is a higher level of abstraction than AND/OR/XOR/NOT gates
  which, in turn, are a higher level of abstraction than transistors
  which, in turn, are a higher level of abstraction than silicon
  regions.  Each layer of abstraction is automatically compiled to a
  lower layer using various tools.

\item[\bf RVI] RISC-V International.  See entry for RISC-V.

\item[\bf SoC] System-on-a-chip.  Refers to a complete computing
  system on a chip, including one or more CPUs (with MMUs and caches),
  shared caches, interconnects, DRAM interface, JTAG, accelerators and
  devices, {\etc}

\item[\bf SRAM] Static Random Access Memory.  A kind of silicon chip
  that implements memory.  See DRAM above for comparison.  Usually
  on-chip in an ASIC or an FPGA.

\item[\bf SystemVerilog] One of the major HDLs.  Originally created in
  the 2000s as a proper superset of Verilog (and thereby subsuming
  Verilog), and incorporating many features from VHDL; incorporated
  some modern features from object-oriented software programming
  languages (principally used in verification testbenches in
  simulation only); then an IEEE standard that has gone through
  several versions.  Can be used for both analog and digital circuits.
  Some features can only be used in simulation (a ``synthesizable
  subset'' can be rendered into hardware).

\item[\bf UART] Universal Asynchronous Receiver/Transmitter.  An
  electronic device attached to a computer system through which the
  CPU can read ASCII characters from a keyboard and send ASCII
  characters to a display screen.  Typically used for the main console
  of a computer system.

\item[\bf Verilog] One of the two grand old HDLs (the other is VHDL).
  Originally created in the 1980s; then an IEEE standard that has gone
  through several versions; then subsumed by SystemVerilog.  Can be
  used for both analog and digital circuits.  Some features can only
  be used in simulation (a ``synthesizable subset'' can be rendered
  into hardware).

\item[\bf VHDL] One of the two grand old HDLs (the other is Verilog).
  Originally created in the 1980s; then an IEEE standard that has gone
  through several versions. Many features were adopted by
  SystemVerilog.  Can be used for both analog and digital circuits.
  Some features can only be used in simulation (a ``synthesizable
  subset'' can be rendered into hardware).

\end{itemize}

% ****************************************************************
