% -*- mode: fundamental -*-

% ****************************************************************

\chapter{RISC-V: the Drum unpipelined CPU (using Rules instead of {\tt StmtFSM})}

\markboth{Ch \arabic{chapter}: The Drum CPU with Rules}{\copyrightnotice}

\setcounter{page}{1}
% \renewcommand{\thepage}{\arabic{page}}
\renewcommand{\thepage}{\arabic{chapter}-\arabic{page}}

\label{ch_Drum_Rules}

% ****************************************************************

\section{Introduction}

In Section~\ref{Sec_Drum_CPU_module_behavior} we showed the complete
behavior of the Drum CPU module, coded using BSV's \verb|StmtFSM|
construct for FSMs.

In this short chapter we show a manual translation of that FSM into
BSV Rules.  Our first objective is to reinforce the fact that
\verb|StmtFSM| does not add any fundamental new computational
capability to BSV---anything that could be coded with \verb|StmtFSM|
can also be coded with Rules.  \verb|StmtFSM| is just a convenient,
more readable view of rules, when the FSM follows a structured flow.

In other situations, where the FSM does not follow a structured flow,
and certainly where we have multiple cooperating FSMs, Rules are often
the more convenient medium of expression.  These ideas are explored in
the exercises the end of the next section.

% ****************************************************************

\section{The Drum CPU module behavior with Rules}

First we define an enumeration type to give symbolic labels to all the
Drum FSM actions:

\SHOWCODE{Code_Extracts/Drum_Rules_Labels.tex}

Then, we instantiate a register \verb|rg_action| to hold the label of
the next action to be performed, initialized to \verb|A_FETCH|.  We
manually transform each FSM action into a rule.  Each rule's explicit
condition checks \verb|rg_action|, and the rule-body updates
\verb|rg_action| to enable the next action.  After the final rule to
execute an exception, we assign \verb|A_FETCH| again to
\verb|rg_action|, and the whole process repeats itself, forever.

\SHOWCODE{Code_Extracts/Drum_Rules.tex}

Please study the FSM version (in
Section~\ref{Sec_Drum_CPU_module_behavior}) and the above code
side-by-side, to see how straightforward it is manually to translate a
\verb|StmtFSM| into Rules.

% ================================================================

\subsection{Optimizing the Drum rules}

If we execute the Drum CPU on a RISC-V program and study the detailed,
cycle-by-cycle execution trace, we will see many possibilities to
eliminate ``wasted'' cycles.

\begin{itemize}

 \item In rule \verb|rl_Decode| we may detect an exception---the IMem
       response itself could be an exception; or Decode may find the
       fetched instruction to be illegal.  In case of an exception,
       instead of proceeding to Register-Read-and-Dispatch, we could
       directly process the exception and return to Fetch.

 \item Similarly, rules \verb|rl_EX_Control|, \verb|rl_EX_Int| may
       raise exceptions; in these cases, too, we could directly deal
       with the exception and return to Fetch.

 \item The Retire Direct CSRRxx action may raise an exception, which
       is handled in the final exception action, which also accesses
       the CSRs (to update trap registers and return the trap-vector
       PC).  These could be fused into a single method in the CSR
       module, after which  we could return directly to Fetch.

 \item Because we are using (a subset of) standard CSRs, the only
       exceptions in a CSRRxx instruction can be because we are
       addressing an unsupported CSR, or we are attempting to update a
       read-only CSR.  Both these questions can be decided in Decode,
       in which case we can assume that, in Retire, the CSRRxx
       treatment will not raise an exeption, and can go straight to
       Fetch.

 \item The Decode and Register-Read-and-Dispatch rules could be
       ``fused'' into a single rule.  Currently, the former rule
       leaves information in a register which is used by the latter
       rule.  If fused, the information can be directly passed as a
       value, and the intervening register could be eliminated.

       Fusing rules together like this comes at a cost: the
       combinational logic in a rule body becomes more complex.  This,
       in turn, may require us to clock it at a slower speed.  Thus,
       while we may have ``saved time'' by eliminating a ``tick'', we
       may have made the ticks longer (slower clock).  Careful
       analysis of the actual numbers is needed to judge whether such
       a transformation is worth it or not.

\end{itemize}

These optimizations are easier to implement in the Rules version of
Drum (instead of the \verb|StmtFSM| version), because the flow now
becomes less structured, with ad-hoc jumps from some rules back to
Fetch, {\etc}.  \verb|StmtFSM| can only express structured,
properly-nested flows.

% ----------------
\hdivider

\Exercise

Implement some or all the ideas above and measure the results (circuit
size, achievable clock speed, application program speed).

\Endexercise
% ----------------

% ****************************************************************

\section{Conclusion}

And that is the complete Drum CPU behavior, written using Rules!  The
reason this was so easy, and why this chapter is so short, is that we
reused all the \verb|Action| definitions defined earlier for the FSM
version of Drum.

% ****************************************************************
