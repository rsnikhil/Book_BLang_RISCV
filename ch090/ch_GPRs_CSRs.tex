% -*- mode: fundamental -*-

% ****************************************************************

\chapter{RISC-V: Modules for GPRs and CSRs}

\markboth{Ch \arabic{chapter}: GPRs and CSRs}{\copyrightnotice}

\setcounter{page}{1}
% \renewcommand{\thepage}{\arabic{page}}
\renewcommand{\thepage}{\arabic{chapter}-\arabic{page}}

\label{ch_GPRs_and_CSRs}

% ****************************************************************

\section{Introduction}

In this chapter we discuss modules for the RISC-V GPRs (General
Purpose Registers) and CSRs (Control and Status Registers).

% ****************************************************************

\section{A register file for RISC-V GPRs, with special treatment of {\tt x0}}

\label{Sec_RISCV_regfile}

\index[RV]{GPRs: general purpose registers}
\index[RV]{x0@{\tt x0}: Special ``always zero'' register}

In RISC-V, register \verb|x0| (index 0) is defined as ``always zero''.
Any value written to \verb|x0| is ignored/discarded, and any read from
\verb|x0| always returns 0.  So, presumably, we do not need an actual
register to hold this value, just some circuitry to ensure that we
always get 0 when we try to ``read'' from \verb|x0|.

In the previous section, we used the module \verb|mkRegFileFull| to
instantiate a register file with 32 registers (inferring 32 from the
full range of the index type \verb|Bit#(5)|).  Instead, we could use
an alternate register file module from the BSV library that allows us
to provide, as module constructor arguments, the lower and upper
indexes of interest.  This instantiates exactly 31 registers indexed
from 1 to 31, thereby saving XLEN bits of register state in our
hardware.

\begin{tabbing}\small\tt
\hmm RegFile \#(Bit \#(5), Bit \#(XLEN)) gprs <- mkRegFile (1, 31);
\end{tabbing}

Regardless of whether we instantiated 31 or 32 registers, RISC-V
instructions can (and do) use {\tt x0} as a source or destination
register, so we need circuitry to deal with attempts to read/write
{\tt x0}.  One possible solution is to make a ``wrapper'' module {\tt
mkRISCV\_GPRs} around the library register file module.

Although we could have reused the {\tt RegFile \#(t1,t2)} interface,
we take the opportunity to define a new interface {\tt
RISCV\_GPRs\_IFC} that has some RISC-V specific method and argument
names, for reading the rs1, rs2 values (register source 1 and 2) and
writing the rd value (register destination):

\index[RV]{RISCV\_GPRs\_IFC@{\tt RISCV\_GPRs\_IFC} interface for {\tt mkRISCV\_GPRs}}

\input{Code_Extracts/RISCV_GPRs.tex}

\index[RV]{mkRISCV\_GPRs@{\tt mkRISCV\_GPRs} a module wrapper around library {\tt RegFile}}

Here is the module implementing the interface:

\input{Code_Extracts/mkRISCV_GPRs.tex}

The module instantiates a library register file {\tt rf}.  The methods
simply invoke the underlying {\tt rf} methods.  The read-methods
override this by returning 0 when the index is 0.

% ----------------
\hdivider

\Exercise
\hm In {\tt mkRISCV\_GPRs} we write the value when \emph{j} is zero,
but we ignore it on reads.  Write a variant where, in {\tt write\_rd},
we alway write 0 when the index {\tt rd} is zero, and the read methods
no longer check if {\tt rs1} or {\tt rs2} are 0.

In this variant, what happens if we try to read {\tt x0} before it is
written?

Compare the circuitry generated in the original and in the variant.
Why might we choose one over the other?

\Exercise
\hm In {\tt mkRISCV\_GPRs} suppose we use {\tt mkRegFile(1,31)} instead of
{\tt mkRegFileFull}.  What needs to change to accommodate this?

\Endexercise

% ================================================================

\subsection{Inlined {\vs} separate module {\tt mkRISCV\_GPRs}}

The above interface and module definitions were parameterized with
``\verb|xlen|'' which is a type-variable (starting with a lower-case
letter).  The interface and module are thus polymorphic in the width
of the data stored in the registers.  Thus, this module can be
instantiated in RV32 and RV64 designs, instantiating \verb|xlen| to 32
and 64, respectively.

Being polymorphic, it will also be inlined wherever it is
instantiated.  If you look at the generated Verilog, there will be no
trace of \verb|mkRISCV_GPRs|; the module code will have been
integrated (inlined) into the parent module's code.

A useful trick is to write a thin, non-polymorphic wrapper for the
module; being non-polymorphic, it can be compiled without inlining
into a distinct Verilog module:

\index[RV]{mkRISCV\_GPRs\_synth@{\tt mkRISCV\_GPRs\_synth} a module wrapper for {\tt mkRISCV\_GPRs} for synthesizability}

\input{Code_Extracts/mkRISCV_GPRs_synth.tex}

Here, we have instantiated the module using \verb|XLEN| which is not a
type-variable, it is defined specifically as 32 or 64 (see
Sec~\ref{BSV_Conditional_compilation}).  The \verb|(*synthesize*)|
attribute can now be respected by the \emph{bsc} compiler and it will
produce a Verilog module \verb|mkRISCV_GPRs_synth|.

% ****************************************************************

\section{A register file for RISC-V CSRs}

\label{Sec_RISCV_CSRs}

The RISC-V CSRs (Control and Status Registers) are not ``just another
register file''.  Here are some significant differences:

\begin{itemize}

 \item There are 32 GPRs, addressed with a 5-bit index.  All of them
       (except one, \verb|x0|) are used in programs.  We can thus use
       a \emph{dense}, packed implementation (\verb|RegFile| from the
       BSV library).

       CSRs are addressed with a 12-bit index (CSR address).  But in
       our implementation we will use just a handful of CSRs
       (\verb|mtvec|, \verb|mtepc|, \verb|mcause|, \verb|mtval| and a
       few more), with non-consecutive addresses, {\ie} the addresses
       are \emph{sparse}; many (most) 12-bit address values are
       unused.

 \item Each GPR (except for \verb|x0|) is ``memory-like''.  When a
       value is written, it remains available for all subsequent reads
       until the next write.  All those reads return the same
       value---the value most recently written.

       CSR reads can, in general, have side effects.  CSR writes, in
       addition to writing a value, can have other side effects as
       well.  A CSR read may not return the same value as the value
       most recently written.

\end{itemize}

For these reasons, we implement each CSR with a separate, ordinary
register.  Here are the CSRs we need for exception-handling:

\input{Code_Extracts/Reg_csr_xxx}

Here are the definitions of standard RISC-V 12-bit addresses for these
CSRs:

\input{Code_Extracts/csr_addrs.tex}

To write a CSR we use a case statement that selects on the CSR
address and writes to the particular CSR.

\input{Code_Extracts/fav_csr_write.tex}

This function returns boolean True on success, and False on failure.
For the moment, failure only means that the argument CSR address was
bad, {\ie} it referred to some unknown CSR.  Failure can also occur if
we try to write to a ``read-only'' CSR (we will see an example later,
CSR TIME).

Similarly, to read a CSR we use a case statement that selects on the
CSR address and reads the particular CSR.

\input{Code_Extracts/fav_csr_read.tex}

This function returns a pair of values (a 2-tuple).  The first
component, like the CSR-write function, is a boolean, True on success
and False on failure.  The second component is the value read from the
CSR.

For the CSRs module we do not export the above read and write
functions directly; they are used internally inside the module.  The
inteface declaration looks like this:

\input{Code_Extracts/CSRs_IFC.tex}

We will discuss these interface methods in more detail when we discuss
trap-handling in Chapter~\ref{ch_Drum_code}.  Suffice it to say, for
now, that:

\begin{tightlist}

 \item the \verb|mav_csrrx| method directly implements the CSRRxx instructions;

 \item the \verb|mav_exception| method directly implements the CSR
       reads and writes needed when taking a trap, and

 \item the \verb|read_epc| method directly reads the MEPC CSR as
       needed by the MRET instruction.

\end{tightlist}
The former was described in Section~\ref{sec_CSRRxx}, and the latter
two were described in Section~\ref{Sec_Traps}.

// ****************************************************************
